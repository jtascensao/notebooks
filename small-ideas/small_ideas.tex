\documentclass[12pt]{article}
\usepackage[hmargin=0.8in,vmargin=0.8in]{geometry}\usepackage[T1]{fontenc}
\usepackage{graphicx}
\usepackage{url}

\setlength{\topskip}{0mm}\setlength{\parskip}{1em}\setlength{\parindent}{0em}
\usepackage{enumitem}\setlist{nolistsep}

\usepackage{natbib}
\setlength{\bibsep}{0.0pt}

\title{Small Ideas}
\author{João Tiago Ascensão}
\date{\today}

\begin{document}
\maketitle
\tableofcontents

\section{Introduction}
\section{Miscellaneous}
\subsection{Creative Anachronism}

In Cryptonomicon \cite{Stephenson2000}, Randy attends the meetings of the Society for Creative Anachronism (SCA), a group founded in UC Berkeley, California, to study and recreate medieval European cultures and their histories before the 17h century.

The \emph{creative} approach differs from historical reenactment.

Creative Anachronism inspires \emph{imagined realities} that borrow selectively from the Middle Ages in combination with fictional elements, including alternate history, fantasy, and sci-fi. 

Examples: Tolkien's novels, \emph{A Song of Ice and Fire} and \emph{Game of Thrones}, role-playing games.

Related ideas: Alternate history, thought fiction.

\bibliographystyle{abbrv}
\bibliography{references}

\end{document}
